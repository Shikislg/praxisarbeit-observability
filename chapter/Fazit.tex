\section{Fazit} 
\subsection{Bedeutung für das Unternehmen}
Da diese Umsetzung sich auf eine Testumgebung beschränkt, entspringt aus der Arbeit für das Unternehmen kein direkter wertschöpfender Nutzen. Die Infrastruktur, die im Rahmen dieser Praxisarbeit aufgebaut wurde, wird aber, sobald sie auf das Produktiv-System übertragen wird, einen großen positiven Einfluss auf die Maintainability der APIs haben. 
\subsection{Bewertung des Erreichens der Zielstellung}
Um das Erreichen der Zielstellung zu evaluieren, werden im Folgenden die vier in den Anforderungen definierten Erwartungen bewertet.
\subsubsection{Erstellung der Datengrundlage}
Die Datengrundlage ist in Form der DynamoDB-Tabelle vollständig funktionsfähig erstellt. Ebenso vollständig funktionsfähig ist die API und das zugehörige Terraform-Backend. Alle angeforderten Funktionen sind hier vollständig abgedeckt
\subsubsection{Datentransport}
Die für den Datentransport benötigte Infrastruktur ist vollständig aufgebaut; Daten können zwischen allen Anwendungen fließen, bei denen ein Datenfluss vorgesehen war. 
\subsubsection{Datenabruf}
Die Lambda-Funktionen, mit denen API-Anfragen automatisch generiert werden, funktionieren fehlerfrei. 
\subsubsection{Daten-Monitoring}
Das DataDog-Dashboard enthält alle in den Anforderungen vorgesehene Funktionen. Insofern sind die Anforderungen erfüllt, wenngleich weitere Entwicklung mit Einbezug von Nutzerfeedback nötig ist, um das Dashboard als vollständig werten zu können.\newline
Da in CloudWatch keine zusätzliche Entwicklung nötig war, ist die Funktionsweise vorhersehbar. In den Anforderungen wurden nur Funktionen beschrieben, die von CloudWatch abgedeckt werden können.
\newline
Insgesamt wurde, gemäß der Zielstellung, eine Infrastruktur erstellt, mit der die Effizienz und Effektivität einer API im AWS-System verbessert werden kann.
\subsection{Ausblick}
Die Übertraung der im Rahmen dieser Praxisarbeit erarbeiteten Infrastruktur auf das Produktiv-System steht in Zukunft im Mittelpunkt. Entsprechende Anpassungen werden vor allem bezüglich der Lambda-Funktion und des Terraform-Backends nötig sein, da die Datenstruktur nicht identisch ist. Erweiterungen der bereits erstellten Infrastruktur können je nach Nutzer-Rückmeldungen durchgeführt werden; erwartet sind diese hauptsächlich im DataDog-Dashboard.y