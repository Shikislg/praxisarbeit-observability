\section{Einleitung und Motivation}
Die Mitarbeiter der Fachabteilungen benutzen eine in AWS aufgebaute API, um auf betriebsrelevante Daten zuzugreifen. Die Reliabilität dieser API ist dementsprechend betrieblich wichtig. Die Hintergrundprozesse in der Datenverarbeitung und Bereitstellung sind jedoch nicht einsichtig; wird eine Information aus einem Datensatz angefordert, lässt sich nur einsehen, ob der Prozess korrekt abgelaufen ist oder nicht. Fehlerdetails werden nicht übermittelt, wodurch die Identifizierung und Behebung von Fehlern nur schwer möglich ist. Zudem ist bisher keine Oberfläche implementiert, mit der die Antwortgeschwindigkeit (im Folgenden ``Response time'') im zeitlichen Kontext überwacht werden kann. Diese beiden Faktoren wirken sich negativ auf die Effizienz und Effektivität der API aus, weshalb die in der Zielsetzung beschriebene Lösung implementiert werden soll.
\subsection{Zielsetzung}
Im Rahmen dieser Praxisarbeit soll die Effizienz  und Effektivität der API gesteigert werden, indem Fehlerinformationen transparent gestaltet werden und die Möglichkeit der Überwachung der Response time geschaffen werden. Für die Dokumentation der Fehler soll die Applikation ``CloudWatch'' Einsicht in die Prozesse geben. Zusätzlich soll die Auftrittshäufigkeit von Fehlern in einem Dashboard dargestellt werden. Die Überwachung der Response time über die Zeit soll durch die Applikation „WatchDog“ erreicht werden. Hierzu sollen relevante Daten zu den jeweiligen Anfragen gesammelt und dargestellt werden, sodass auch Trends erkennbar sind. 
\subsection{Aufbau der Arbeit}
Zuerst sollen die Konzepte und Werkzeuge geklärt werden, mit denen in der Arbeit gearbeitet wird. Dann sollen die konkreten Anforderungen definiert werden, die aus der Zielsetzung hervorgehen. Dann soll die Umsetzung dieser Anforderungen erläutert werden.  
\subsection{Unternehmen}
Bayer, gegründet 1863 als „Friedr. Bayer et comp.“ , mit Sitz in Leverkusen ist das weltweit umsatzstärkste Unternehmen im Bereich CropScience. Aufgeteilt ist das Unternehmen in drei Divisionen: Pharmaceuticals, also der Herstellung verschreibungspflichtiger Arzneimittel, Consumer Health, die Herstellung sogenannter „Over-the-Counter Medikamente“, für welche keine ärztliche Verschreibung nötig ist, und CropScience, der Pflanzenschutzdivision. Zusätzlich zu den drei Divisionen existieren die „Enabling Functions“. Diese unterstützen das Tagesgeschäft, sind aber keiner der Divisionen direkt untergeordnet, sondern stehen neben den Divisionen und unterstützen diese. Diese Praxisarbeit behandelt ein Projekt, welches primär für Crop-Science Nutzer relevant ist, jedoch im Auftrag einer Abteilung der „Enabling Functions“ durchgeführt wird. Für die Enabling Function ist Effizienz und Effektivität ein wichtiges Kriterium, woraus sich die Relevanz dieser Arbeit ableitet.
\subsection{Relevanz für das Unternehmen}
Neben den geringeren Prozesskosten in der Verantwortung der Enabling Functions führt eine gesteigerte Effizienz und Effektivität zu einer verbesserten Nutzerfreundlichkeit. Durch Benachrichtigungen können Fehler schneller behoben werden, wodurch die Uptime erhöht wird und durch Monitoring der Response time kann entsprechend bei negativen Trends dieser früher gehandelt werden. Dadurch soll die API schlussendlich, gestützt von den zu implementierenden Anwendungen, zuverlässiger laufen und zu verbessertem Nutzerempfinden führen.
\subsection{Einschränkung}
Aus zeitlichen Gründen wird im Rahmen dieser Praxisarbeit die Implementierung nur auf einem Test-System durchgeführt, und alle beschriebenen Komponenten sind nur für Testzwecke aufgebaut. Die Daten haben dementsprechend nicht immer semantische Relevanz, solange eine solche keinen Einfluss auf die Feststellung der Funktionalität der Infrastruktur haben. Die Anwendungen, mit denen im Rahmen dieser Praxisarbeit gearbeitet wird, sind Teil der Standard-Suite der Abteilung, die den Auftrag für diese Arbeit erteilt hat. Es hat keine eigene Auswahl bei diesen Anwendungen stattgefunden, weshalb die Auswahl der Anwendungen auch kein Thema in dieser Arbeit ist.